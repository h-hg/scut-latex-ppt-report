\documentclass[
  UTF8,
  twoside,
  zihao=5,
  scheme=plain,
  heading=true,
]{ctexart}

\usepackage{geometry}
\geometry{
    a4paper,
    %margin=15mm,
    top=1.5cm,
    bottom=1.5cm,
    left=1.5cm,
    right=1.5cm
}

\usepackage{indentfirst}

\title{基于分治的数据驱动的大规模演化算法 - 演讲稿}
% \author{黄浩淦}
% \date{2023 年 10 月 12 日}

\begin{document}

\maketitle

\section{Introduction}

首先,我先给出优化问题的定义,即对一个有着 $d$ 维的目标函数求最小值。
但通常这个目标是不可导,并且可能是多峰的,也有可能是黑盒子优化问题。
演化算法作为一种启发式算法,在很多科学上、运筹上、工程上的优化问题都得到了很好的表型。
演化演化算法大多都基于一个假设,即对候选解的评估是简单的、廉价的
然后现实生活中的许多评估都是昂贵的。
例如在音乐的生成领域,对音乐的奖赏通常需要人这一主体的参与,这通常是代价昂贵;
在高保真的模拟实验中,对解的评估通常需要运行一遍模拟程序,这通常是耗时长的;
此外,有些优化问题的解是有危险的,例如药物研发等。

为了解决代价高的演化算法的困境,研究学者提出了一种基于代理模型的演化算法,它从历史已有的评价过的数据入手,从这些数据中构建代理模型,用以替换演化算法迭代过程中部分或者全部的适应值评估。
由于代理模型是从数据中构建而来,所以这种优化算法也被称为数据驱动的演化算法。

根据真实的适应值评估函数在优化过程中是否再次使用到,可以将数据驱动的演化算法分为两类,一类是在线的数据驱动演化算法,这一类在迭代优化的过程中依旧可以使用少量的真实适应值评估函数,另一类被称为离线的数据驱动演化算法,在迭代的过程中无法再次使用到真实的适应值评估函数。
在使用代理模型的优化的过程中,代理模型不是一成不变的,而是会作出调整。
例如在在线驱动的演化算法中,会使用真实的适应值评估函数去评估预测值比较高的个体或者预测值不确定的个体,并将过这些评测后的样本重新加入到训练集中,重新训练模型以调整模型效果;
在离线的数据驱动中,由于无法再次使用真实的适应值评估,所以一般是通过调整数据的分布来调整模型。

下面我介绍什么是大规模优化问题,简单的说就是决策变量很多,甚至高达上千的优化问题。
大规模优化问题给数据驱动的演化算法带来了很多新的挑战
\begin{enumerate}
  \item 搜索空呈指数级增长,让搜索空间的地貌特征发生了改变,例如单峰地貌变成了多峰地貌。
  \item 训练代理模型的难度急剧上升,随着维度的增加,也引发了“维度灾难”的问题。
  例如 Krigin 是一个常见的代理模型,有着比较高的预测精度,但由于其复杂度较大,在高维的空间上,使得 Krigin 模型的空间复杂度和时间复杂度急剧上升。
  \item 数据驱动的优化问题通常是昂贵的或耗时的优化问题,在这种模式下,能采集到的数据十分少。
  在巨大的搜索空间中,一个由代理模型引入的小误差都可能误导进化算法的搜索方向。
  因此,通过使用有限的数据建立代理模型在巨大的搜索空间上进行搜索,其挑战十分巨大。
\end{enumerate}

因此,目前并没有许多关于数据驱动的大规模演化算法的研究。


前人在解决大规模优化问题可以分成四类,一是降维,二是增强数据,三是使用更为先进的演化算法,四是基于分而治之。
本次的研究也是基于分而治之实现的。

\section{Proposal Algorithms}

首先是整个算法的流程,在演化算法的每一代中,首选会判断是否需要对空间进行重新划分,如果需要的话,则产生新的划分规则,并训练新的代理模型,种群会依照代理模型而划分为多个子种群,在这种情况下,可以使用并行的方式对子种群进行优化,在子种群的优化过程中。
我使用了两种梯度算子,一是使用 EA 算法进行探索,二是使用基于梯度更新的算子进行开发。
最后使用两个合并策略完成子种群的信息汇聚。
不断重复这个过程,直到所有达到解决条件。

首先要介绍的是本研究中提出一种全空间-子空间联合学习的代理模型,它包括一个全局模型以及多个局部代理模型。
局部代理模型可以在于预测子个体在子空间上的适应值,而全局模型可以用来个体在全空间的适应值。
每次局部代理模型都是使用离线数据相应维度的数据来进行构建,而全局模型由局部模型线性组成,其权重与局部代理模型的 MSE 相关。

在空间的划分上,我采用了一种随机划分的策略,这是因为在离线优化中,无法使用真实的适应值评估函数来获得决策变量之间的关系。
此外,还提出了一个动态空间划分策略,简单来说,有两个步骤。
1. 每隔 $T_g$ 代,就会将子空间的数量减去 1,这也满足前期探索后期开发的特点。
2. 在子空间数量不变的情况下,每隔 $T_r$ 代就会对子空间重新划分,增加具有相关性的决策变量在同一空间中被优化的概率。

在离线数据驱动的演化中,由于真实的适应值评估函数无法被再次使用,可以近似地将代理模型的预测值作为目标函数。从另一个角度上看,这相当于将代理模板看作目标函数,进而求解其最小值。
在传统的演化算法中,目标函数大多数是不可微分的,而代理模型通过是通过机器学习等方法获得的,具有可微分性。
从公式 7 可以看出,全局代理模型对子种群的个体的微分可转化为对局部代理模型对子个体的优化,进而使得梯度算子可以并行优化。
同时,我也还是用了 Admam 优化器来调整梯度算子的学习率。
在高维的空间,一个小的改变都可能使得个体的适应值发生巨大的改变,使用基于梯度更新的方式,它每次移动的步长十分的小,可以有效减低个体错过最优值的机会。

最后是种群的更新方式,我采用了两个策略。
一是让子种群按照代理模型的预测值进行排序,然后按照排序进行合并;
二是让子种群中适应值高的个体随机合并。
策略二可以很好地保存最优子问题的解在后代中的比例。

\section{Experiments}

首先给出本研究中一些超参数的取值,其中迭代次数为 100,局部代理模型为 RBFN,EA 算子使用的是差分演化算子。

首先,我将所提出的算法 CC-DDEA,与五个 SOTA 的离线数据驱动算法进行比较,从表中可以看出 CC-DDEA 几乎在所有 benchamark 上都取得了最好的成绩。
左边是六种算法在时间上的对比,虽然 CC-DDEA 在时间上并没有取得最好,但是由于并行优化的使用,CC-DDEA 对维度敏感度较低。
右边是六种算法的收敛曲线图,可以看出,一看是 CC-DDEA 收敛速度并不是很快,随着子空间的数量逐渐减少,CC-DDEA 的开发程度变大,收敛速度加快。

此外,CC-DDEA 还与五个用于处于高维的在线数据驱动算法进行比较,CC-DDEA 表现出了优秀的性能。

接着,是消融实验,我选了四个变体,分别是使用普通的 RBFN 代替提出的 HSJL、使用随机合并的策略、使用 DE 算子、删除梯度算子,从表中可以中可以看出,每个变体都不如 CC-DDEA,这说明,所提出的模块都对整体有促进作用。

然后,是参数敏感性分析。
第一个需要分析的是初始子空间的数量 $g_{init}$ 以及影响子空间变化的 $T_g$。
我定义一个变化速率 R,表示子空间数量变为 1 所需要的代数占比,并选取了多个速率以及初始子空间数量。
实验表明,当 $g_{init} = 10$ 以及 $T_g = 8$ 的时候,可以最优的结果。
第二个需要分析的是 $T_r$,它表示在子空间数量不变的情况下,代理模型更新的频率,我选择了三个值,分别为只更新一次,更新两次,每次都更新。
结果表明只更新一次的效果最差,每次都更新与更新两次的效果差不多,但是每次都更新,所占用的时间开销比较多。
所以我选了更新两次这种方案。

最后,我将 CC-DDEA 运用到一个交通灯优化的现实优化中,在该场景中,需要优化为的决策变量高达 500 多维,并比较了多个离线的数据驱动算法优化的结果,从箱线图可以看出,CC-DDEA 取得了最好的效果。

\section{Conclusion}

1. 提出了一种全空间-子空间联合学习的代理模型,它包含了一个全局模型以及多个局部模型。
2. 采用了分而治之的策略,同时采用了一种动态空间划分策略,很好地平衡了开发与探索。
3. 在子空间优化中,提出了一种基于梯度算子的更新策略。
4. 提出了一种合并子群的策略,可以很好地保存子问题的优秀解。

\end{document}